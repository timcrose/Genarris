%% Generated by Sphinx.
\def\sphinxdocclass{report}
\documentclass[letterpaper,10pt,english]{sphinxmanual}
\ifdefined\pdfpxdimen
   \let\sphinxpxdimen\pdfpxdimen\else\newdimen\sphinxpxdimen
\fi \sphinxpxdimen=.75bp\relax

\PassOptionsToPackage{warn}{textcomp}
\usepackage[utf8]{inputenc}
\ifdefined\DeclareUnicodeCharacter
% support both utf8 and utf8x syntaxes
  \ifdefined\DeclareUnicodeCharacterAsOptional
    \def\sphinxDUC#1{\DeclareUnicodeCharacter{"#1}}
  \else
    \let\sphinxDUC\DeclareUnicodeCharacter
  \fi
  \sphinxDUC{00A0}{\nobreakspace}
  \sphinxDUC{2500}{\sphinxunichar{2500}}
  \sphinxDUC{2502}{\sphinxunichar{2502}}
  \sphinxDUC{2514}{\sphinxunichar{2514}}
  \sphinxDUC{251C}{\sphinxunichar{251C}}
  \sphinxDUC{2572}{\textbackslash}
\fi
\usepackage{cmap}
\usepackage[T1]{fontenc}
\usepackage{amsmath,amssymb,amstext}
\usepackage{babel}



\usepackage{times}
\expandafter\ifx\csname T@LGR\endcsname\relax
\else
% LGR was declared as font encoding
  \substitutefont{LGR}{\rmdefault}{cmr}
  \substitutefont{LGR}{\sfdefault}{cmss}
  \substitutefont{LGR}{\ttdefault}{cmtt}
\fi
\expandafter\ifx\csname T@X2\endcsname\relax
  \expandafter\ifx\csname T@T2A\endcsname\relax
  \else
  % T2A was declared as font encoding
    \substitutefont{T2A}{\rmdefault}{cmr}
    \substitutefont{T2A}{\sfdefault}{cmss}
    \substitutefont{T2A}{\ttdefault}{cmtt}
  \fi
\else
% X2 was declared as font encoding
  \substitutefont{X2}{\rmdefault}{cmr}
  \substitutefont{X2}{\sfdefault}{cmss}
  \substitutefont{X2}{\ttdefault}{cmtt}
\fi


\usepackage[Sonny]{fncychap}
\ChNameVar{\Large\normalfont\sffamily}
\ChTitleVar{\Large\normalfont\sffamily}
\usepackage{sphinx}

\fvset{fontsize=\small}
\usepackage{geometry}

% Include hyperref last.
\usepackage{hyperref}
% Fix anchor placement for figures with captions.
\usepackage{hypcap}% it must be loaded after hyperref.
% Set up styles of URL: it should be placed after hyperref.
\urlstyle{same}

\usepackage{sphinxmessages}
\setcounter{tocdepth}{1}



\title{Genarris}
\date{Sep 17, 2019}
\release{2.0}
\author{Rithwik Tom, Timothy Rose, Imanuel Bier}
\newcommand{\sphinxlogo}{\vbox{}}
\renewcommand{\releasename}{Release}
\makeindex
\begin{document}

\ifdefined\shorthandoff
  \ifnum\catcode`\=\string=\active\shorthandoff{=}\fi
  \ifnum\catcode`\"=\active\shorthandoff{"}\fi
\fi

\pagestyle{empty}
\sphinxmaketitle
\pagestyle{plain}
\sphinxtableofcontents
\pagestyle{normal}
\phantomsection\label{\detokenize{index::doc}}



\chapter{How to use API Documentation}
\label{\detokenize{index:how-to-use-api-documentation}}
The execution of Genarris is controlled by a \sphinxhref{https://docs.python.org/3.4/library/configparser.html}{configuration} file. The configuration
file specifies the execution of Genarris which is broken down into procedures,
such as \sphinxcode{\sphinxupquote{Pygenarris\_Structure\_Generation}}. Each procedure has a corresponding
section in the configuration file, for our example \sphinxcode{\sphinxupquote{pygenarris\_structure\_generation}}.
The section contains options which control the operations performed by each
procedure.

This document details the options for procedures that are executed in the Genarris 2.0
\sphinxstyleemphasis{Robust} workflow. In order these are, \sphinxcode{\sphinxupquote{Relax\_Single\_Molecule, Estimate\_Unit\_Cell\_Volume,
Pygenarris\_Structure\_Generation, Run\_Rdf\_Calc, Affinity\_Propagation\_Fixed\_Clusters,
FHI\_Aims\_Energy\_Evaluation, Affinity\_Propagation\_Fixed\_Clusters, Run\_FHI\_Aims\_Batch}}.
There are many options that can be specified and modified for each section.
All of these options are specified in this document under the
\sphinxstylestrong{Configuration File Options} section of each procedure.

There are three categories of \sphinxstylestrong{Configuration File Options}. These are \sphinxstyleemphasis{required},
\sphinxstyleemphasis{optional}, and \sphinxstyleemphasis{inferred}. In the \sphinxstylestrong{Configuration File Options}, these categories
are specified after the \sphinxstyleemphasis{type} of the option, such as \sphinxstyleemphasis{int}, \sphinxstyleemphasis{float}, or \sphinxstyleemphasis{bool}.
\sphinxstyleemphasis{Required} options have no category placed after the type. Both \sphinxstyleemphasis{optional} and
\sphinxstyleemphasis{inferred} are specified after the type. \sphinxstyleemphasis{Optional} arguments are those that
have default settings that in general perform well. The user may specify these
\sphinxstyleemphasis{optional} arguments to have more control over the program executing.
\sphinxstyleemphasis{Inferred} options are those that may be present in multiple different procedures.
For example, the option \sphinxcode{\sphinxupquote{aims\_lib\_dir}} is needed in the \sphinxcode{\sphinxupquote{Relax\_Single\_Molecule}},
\sphinxcode{\sphinxupquote{FHI\_Aims\_Energy\_Evaluation}}, and \sphinxcode{\sphinxupquote{Run\_FHI\_Aims\_Batch}}. But, because it is
an inferred parameter, it only needs to be specified once in the earliest procedure
in which occurs and then it will be inferred by all further procedures. Options which
are inferred are thus optional in all proceeding sections.


\chapter{Genarris 2.0 Procedures for Robust Workflow}
\label{\detokenize{index:genarris-2-0-procedures-for-robust-workflow}}\index{Genarris (class in Genarris.genarris\_master)@\spxentry{Genarris}\spxextra{class in Genarris.genarris\_master}}

\begin{fulllineitems}
\phantomsection\label{\detokenize{index:Genarris.genarris_master.Genarris}}\pysiglinewithargsret{\sphinxbfcode{\sphinxupquote{class }}\sphinxcode{\sphinxupquote{Genarris.genarris\_master.}}\sphinxbfcode{\sphinxupquote{Genarris}}}{\emph{inst\_path}}{}
Master class of Genarris. It controls all aspects of the
Genarris workflow which can be executed individually or sequantially.
Begins by reading and intepreting the configuration file.
Calls the defined procedures with the options specified in the
configuration file. Some options may be inferred from previous sections
if they are not present in every section.
\begin{description}
\item[{Arguments}] \leavevmode\begin{description}
\item[{inst\_path: str}] \leavevmode
Path to the configuration file.

\end{description}

\end{description}
\index{Affinity\_Propagation\_Fixed\_Clusters() (Genarris.genarris\_master.Genarris method)@\spxentry{Affinity\_Propagation\_Fixed\_Clusters()}\spxextra{Genarris.genarris\_master.Genarris method}}

\begin{fulllineitems}
\phantomsection\label{\detokenize{index:Genarris.genarris_master.Genarris.Affinity_Propagation_Fixed_Clusters}}\pysiglinewithargsret{\sphinxbfcode{\sphinxupquote{Affinity\_Propagation\_Fixed\_Clusters}}}{\emph{comm}}{}
AP that explores the setting of preference in order to generate
desired number of clusters.
\begin{quote}\begin{description}
\item[{Parameters}] \leavevmode
\sphinxstylestrong{comm} (\sphinxstyleemphasis{mpi4py.MPI object}) -- MPI communicator.

\end{description}\end{quote}
\subsubsection*{Configuration File Options}
\begin{description}
\item[{output\_dir}] \leavevmode{[}str{]}
Path to the directory where the chosen structures will be stored.

\item[{preference\_range}] \leavevmode{[}list{]}
List of two values as the {[}min, max{]} of the range of allowable
preference values.

\item[{structure\_dir}] \leavevmode{[}str, inffered{]}
Path to the directory of files to be used for the calculation.
Default is to infer this value from the previous section.

\item[{dist\_mat\_input\_file}] \leavevmode{[}str, inferred{]}
Path to the distance matrix output from the descriptor calculation.
Default is to infer this value from the previous sections.

\item[{output\_format}] \leavevmode{[}str, optional{]}
Format the structure files should be saved as. Default is both.

\item[{cluster\_on\_energy}] \leavevmode{[}bool, optional{]}
Uses energy values to determine examplars. Structures with the
lowest energy values from each cluster are selected.
Default is False.

\item[{plot\_histograms}] \leavevmode{[}bool, optional{]}
If histogram plots should be created of the volume and space
groups. Default is False.

\item[{num\_of\_clusters}] \leavevmode{[}int or float, optional{]}
Float, must be less than 0. Selects a fraction of the structures.
Int, selects specific number of structures equal to int.
Default is 0.1.

\item[{num\_of\_clusters\_tolerance}] \leavevmode{[}int, optional{]}
Algorithm will stop if it has generated the number of clusters
within the number of desired clusters and this tolerance.
Default is 0.

\item[{max\_sampled\_preferences}] \leavevmode{[}int, optional{]}
Maximum number of preference values to try.

\item[{output\_without\_success}] \leavevmode{[}bool, optional{]}
Whether to perform output procedures if the algorithm has reached
the maximum number of sampled preferences without finding the
correct number of clusters. Default is False.

\item[{affinity\_type}] \leavevmode{[}list, optional{]}
List of {[}type of afinity, value{]} argument Scikit-Learn AP alogrithm.

\item[{affinity\_matrix\_path}] \leavevmode{[}str, optional{]}
Path to the affinity matrix to use for the AP algorithm.
Default is "affinity\_matrix.dat".

\item[{damping}] \leavevmode{[}float, optional{]}
damping argument for Scikit-Learn AP algorithm. Default is 0.5.

\item[{convergence\_iter}] \leavevmode{[}int, optional{]}
convergence\_iter argument for Scikit-Learn AP algorithm.
Default is 15.

\item[{max\_iter}] \leavevmode{[}int, optional{]}
max\_iter argument for Scikit-Learn AP algorithm. Default is 1000.

\item[{preference}] \leavevmode{[}int, optional{]}
preference argument for Scikit-Learn AP algorithm. Default is None.

\item[{verbose\_output}] \leavevmode{[}bool, optional{]}
verbose argument for Scikit-Learn AP algorithm. Default is False.

\item[{property\_key}] \leavevmode{[}str, optional{]}
Key which the AP cluster will be stored in the properties of
each structure object. Default is AP\_cluster.

\item[{output\_file}] \leavevmode{[}str, optional{]}
Path where info about the AP alogrithm execution will be stored.
Default is "./AP\_cluster.info".

\item[{exemplars\_output\_dir}] \leavevmode{[}str, optional{]}
If provided, will output the examplars of each cluster to this
folder. Default is None.

\item[{exemplars\_output\_format}] \leavevmode{[}str, optional{]}
File format of structures to be output. Default is both.

\item[{structure\_suffix}] \leavevmode{[}str, optional{]}
Suffix to apply to structure files which are written.
Default is ".json".

\end{description}

\end{fulllineitems}

\index{Estimate\_Unit\_Cell\_Volume() (Genarris.genarris\_master.Genarris method)@\spxentry{Estimate\_Unit\_Cell\_Volume()}\spxextra{Genarris.genarris\_master.Genarris method}}

\begin{fulllineitems}
\phantomsection\label{\detokenize{index:Genarris.genarris_master.Genarris.Estimate_Unit_Cell_Volume}}\pysiglinewithargsret{\sphinxbfcode{\sphinxupquote{Estimate\_Unit\_Cell\_Volume}}}{\emph{comm}}{}
Performs volume estimation using a machine learned model train on the
CSD and based on Monte Carlo volume integration and topological
molecular fragments. See Genarris 2.0 paper for full description.
\begin{quote}\begin{description}
\item[{Parameters}] \leavevmode
\sphinxstylestrong{comm} (\sphinxstyleemphasis{mpi4py.MPI object}) -- MPI communicator.

\end{description}\end{quote}
\subsubsection*{Configuration File Options}
\begin{description}
\item[{volume\_mean}] \leavevmode{[}float, optional{]}
If provided, uses this value as the volume generation mean without
using the ML model to etimate the volume.

\item[{volume\_std}] \leavevmode{[}float, optional{]}
If provided, uses this value for structure generation, otherwise
a default value of 0.075 is provided.

\end{description}
\begin{quote}\begin{description}
\item[{Returns}] \leavevmode
\sphinxstylestrong{None} (\sphinxstyleemphasis{None}) -- Returns an object of type None.

\end{description}\end{quote}

\end{fulllineitems}

\index{FHI\_Aims\_Energy\_Evaluation() (Genarris.genarris\_master.Genarris method)@\spxentry{FHI\_Aims\_Energy\_Evaluation()}\spxextra{Genarris.genarris\_master.Genarris method}}

\begin{fulllineitems}
\phantomsection\label{\detokenize{index:Genarris.genarris_master.Genarris.FHI_Aims_Energy_Evaluation}}\pysiglinewithargsret{\sphinxbfcode{\sphinxupquote{FHI\_Aims\_Energy\_Evaluation}}}{\emph{comm}, \emph{world\_comm}, \emph{MPI\_ANY\_SOURCE}, \emph{num\_replicas}}{}
Runs Self-Consistent Field calculation on a pool of structures.
\begin{quote}\begin{description}
\item[{Parameters}] \leavevmode
\sphinxstylestrong{See :meth:{}`Run\_FHI\_Aims\_Batch{}`}

\end{description}\end{quote}
\subsubsection*{Configuration File Options}

See {\hyperref[\detokenize{index:Genarris.genarris_master.Genarris.Run_FHI_Aims_Batch}]{\sphinxcrossref{\sphinxcode{\sphinxupquote{Run\_FHI\_Aims\_Batch()}}}}}
\begin{quote}\begin{description}
\item[{Returns}] \leavevmode
\sphinxstylestrong{None} (\sphinxstyleemphasis{None})

\end{description}\end{quote}

\end{fulllineitems}

\index{Pygenarris\_Structure\_Generation() (Genarris.genarris\_master.Genarris method)@\spxentry{Pygenarris\_Structure\_Generation()}\spxextra{Genarris.genarris\_master.Genarris method}}

\begin{fulllineitems}
\phantomsection\label{\detokenize{index:Genarris.genarris_master.Genarris.Pygenarris_Structure_Generation}}\pysiglinewithargsret{\sphinxbfcode{\sphinxupquote{Pygenarris\_Structure\_Generation}}}{\emph{comm}}{}
Uses the Genarris module written in C to perform structure generation.
This module enables generation on special positions.
\begin{quote}\begin{description}
\item[{Parameters}] \leavevmode
\sphinxstylestrong{comm} (\sphinxstyleemphasis{mpi4py.MPI object}) -- MPI communicator.

\end{description}\end{quote}
\subsubsection*{Configuration File Options}
\begin{description}
\item[{molecule\_path}] \leavevmode{[}str{]}
Path to the relaxed molecule geometry.

\item[{output\_format}] \leavevmode{[}str,{]}
Determines the type of file which will be output for each
structure. Can be one of: json, geo, both.

\item[{output\_dir}] \leavevmode{[}str{]}
Path to the directory which will contain all generated structures
which pass the intermolecular distance checks.

\item[{num\_structures}] \leavevmode{[}int{]}
Target number of structures to generate.

\item[{Z}] \leavevmode{[}int{]}
Number of molecules per cell to generate.

\item[{volume\_mean}] \leavevmode{[}float, optional{]}
See {\hyperref[\detokenize{index:Genarris.genarris_master.Genarris.Estimate_Unit_Cell_Volume}]{\sphinxcrossref{\sphinxcode{\sphinxupquote{Estimate\_Unit\_Cell\_Volume()}}}}}

\item[{volume\_std}] \leavevmode{[}float, optional{]}
See {\hyperref[\detokenize{index:Genarris.genarris_master.Genarris.Estimate_Unit_Cell_Volume}]{\sphinxcrossref{\sphinxcode{\sphinxupquote{Estimate\_Unit\_Cell\_Volume()}}}}}

\item[{sr}] \leavevmode{[}float, optional{]}
Defines the minimum intermolecular distance that is considered
physical by multiplying the sum of the van der Waals radii of the
interacting atoms by sr. Default value is 0.85.

\item[{tol}] \leavevmode{[}float, optional{]}
Tolerance to be used to identify space groups compatible with the
input molecule.

\item[{max\_attempts\_per\_spg\_per\_rank}] \leavevmode{[}int{]}
Defines the maximum number of attempts the structure generator
makes before moving on to the next space group.

\item[{num\_structures\_per\_allowed\_SG\_per\_rank}] \leavevmode{[}int{]}
Number of structures per space group per rank which will be
generated by Pygenarris.

\item[{geometry\_out\_filename}] \leavevmode{[}str{]}
Filename where all structures generated by Pygenarris will be found.

\item[{omp\_num\_threads}] \leavevmode{[}int{]}
Number of OpenMP threads to pass into Pygenarris

\item[{truncate\_to\_num\_structures}] \leavevmode{[}bool{]}
If true, will reduce pool to exactly the number defined by
num\_structures.

\end{description}

\end{fulllineitems}

\index{Relax\_Single\_Molecule() (Genarris.genarris\_master.Genarris method)@\spxentry{Relax\_Single\_Molecule()}\spxextra{Genarris.genarris\_master.Genarris method}}

\begin{fulllineitems}
\phantomsection\label{\detokenize{index:Genarris.genarris_master.Genarris.Relax_Single_Molecule}}\pysiglinewithargsret{\sphinxbfcode{\sphinxupquote{Relax\_Single\_Molecule}}}{\emph{comm}, \emph{world\_comm}, \emph{MPI\_ANY\_SOURCE}, \emph{num\_replicas}}{}
Calls run\_fhi\_aims\_batch using the provided single molecule path.
\begin{quote}\begin{description}
\item[{Parameters}] \leavevmode
\sphinxstylestrong{for\_parameters} -- See {\hyperref[\detokenize{index:Genarris.genarris_master.Genarris.Run_FHI_Aims_Batch}]{\sphinxcrossref{\sphinxcode{\sphinxupquote{Run\_FHI\_Aims\_Batch()}}}}}

\end{description}\end{quote}
\subsubsection*{Configuration File Options}

See {\hyperref[\detokenize{index:Genarris.genarris_master.Genarris.Run_FHI_Aims_Batch}]{\sphinxcrossref{\sphinxcode{\sphinxupquote{Run\_FHI\_Aims\_Batch()}}}}}
\begin{quote}\begin{description}
\item[{Returns}] \leavevmode
\sphinxstylestrong{None} (\sphinxstyleemphasis{None}) -- Returns an object of type None.

\end{description}\end{quote}

\end{fulllineitems}

\index{Run\_FHI\_Aims\_Batch() (Genarris.genarris\_master.Genarris method)@\spxentry{Run\_FHI\_Aims\_Batch()}\spxextra{Genarris.genarris\_master.Genarris method}}

\begin{fulllineitems}
\phantomsection\label{\detokenize{index:Genarris.genarris_master.Genarris.Run_FHI_Aims_Batch}}\pysiglinewithargsret{\sphinxbfcode{\sphinxupquote{Run\_FHI\_Aims\_Batch}}}{\emph{comm}, \emph{world\_comm}, \emph{MPI\_ANY\_SOURCE}, \emph{num\_replicas}}{}
Runs FHI-aims calculations on a pool of structures using num\_replicas.
\begin{quote}\begin{description}
\item[{Parameters}] \leavevmode\begin{itemize}
\item {} 
\sphinxstylestrong{comm} (\sphinxstyleemphasis{mpi4py.MPI object}) -- MPI communicator to pass into aims

\item {} 
\sphinxstylestrong{world\_comm} (\sphinxstyleemphasis{mpi4py.MPI object}) -- World MPI communicator

\item {} 
\sphinxstylestrong{MPI\_ANY\_SOURCE} (\sphinxstyleemphasis{mpi4py.MPI.ANY\_SOURCE}) -- MPI ANY\_SOURCE object to facilitate communication.

\item {} 
\sphinxstylestrong{num\_replicas} (\sphinxstyleemphasis{int}) -- Number of replicas to use in calculation.

\end{itemize}

\end{description}\end{quote}
\subsubsection*{Configuration File Options}
\begin{description}
\item[{verbose}] \leavevmode{[}bool{]}
Controls verbosity of output.

\item[{energy\_name}] \leavevmode{[}str{]}
Property name which the calculated energy will be stored with in the
Structure file.

\item[{output\_dir}] \leavevmode{[}str{]}
Path to the directory where the output structure file will be saved.

\item[{aims\_output\_dir}] \leavevmode{[}str{]}
Path where the aims calculation will take place.

\item[{aims\_lib\_dir}] \leavevmode{[}str{]}
Path to the location of the directory containing the FHI-aims library
file.

\item[{molecule\_path}] \leavevmode{[}str{]}
Path to the geometry.in file of the molecule to be calculated if
called using harris\_single\_molecule\_prep or relax\_single\_molecule.

\item[{structure\_dir}] \leavevmode{[}str{]}
Path to the directory of structures to be calculated if calculation
was called not using harris\_single\_molecule\_prep or
relax\_single\_molecule.

\end{description}
\begin{quote}\begin{description}
\item[{Returns}] \leavevmode
\sphinxstylestrong{None} (\sphinxstyleemphasis{None})

\end{description}\end{quote}

\end{fulllineitems}

\index{Run\_Rdf\_Calc() (Genarris.genarris\_master.Genarris method)@\spxentry{Run\_Rdf\_Calc()}\spxextra{Genarris.genarris\_master.Genarris method}}

\begin{fulllineitems}
\phantomsection\label{\detokenize{index:Genarris.genarris_master.Genarris.Run_Rdf_Calc}}\pysiglinewithargsret{\sphinxbfcode{\sphinxupquote{Run\_Rdf\_Calc}}}{\emph{comm}}{}
Runs RDF calculation for the pool of generated structures. RDF
descriptor is similar to that described in Behler and Parrinello 2007.
Then calculates the structure difference matrix.
\begin{quote}\begin{description}
\item[{Parameters}] \leavevmode
\sphinxstylestrong{comm} (\sphinxstyleemphasis{mpi4py.MPI object}) -- MPI communicator.

\end{description}\end{quote}
\subsubsection*{Configuration File Options}
\begin{description}
\item[{dist\_mat\_fpath}] \leavevmode{[}str{]}
Path to file to write distance matrix to.

\item[{output\_dir}] \leavevmode{[}str{]}
Path of directory to write structures to (will create if it DNE).
If 'no\_new\_output\_dir' then input structures will be overwritten.

\item[{normalize\_rdf\_vectors: bool,optional}] \leavevmode
Whether to normalize the rdf vectors over the columns of the
feature matrix before using them to compute the distance matrix.
Default is Falase.

\item[{standardize\_distance\_matrix: bool}] \leavevmode
If True, standardizes the distance matrix. The method is to divide
all elements by the max value in the distance matrix.
Because it is a distance matrix and thus all elements are positive,
the standardized elements will be in the range {[}0, 1{]}.
Default is False.

\item[{save\_envs: bool, optional}] \leavevmode
Whether to save the environment vectors calculated by the RDF
method in the output structure files. Default is False.

\item[{cutoff}] \leavevmode{[}float, optional{]}
Cutoff radius to apply to the atom centered symmetry function.
Default is 12.

\item[{n\_D\_inter}] \leavevmode{[}int, optional{]}
Number of dimensions to use for each type of pair-wise
interatomic interaction found in the structure. Default is 12.

\item[{init\_scheme}] \leavevmode{[}str, optional{]}
Can be centered or shifted, as described in Gastegger et al. 2018.
Default is shifted.

\item[{eta\_range}] \leavevmode{[}list, optional{]}
List of two floats which define the range for eta parameter in
Gastegger et al. 2018. Default is {[}0.05,0.5{]}.

\item[{Rs\_range}] \leavevmode{[}list, optional{]}
List of two floats which define the range for Rs parameter in
Gastegger et al. 2018. Default is {[}{[}0.1,12{]}.

\item[{pdist\_distance\_type}] \leavevmode{[}str,optional{]}
Input parameter for the pdist function. Default is Euclidean.

\end{description}
\begin{quote}\begin{description}
\item[{Returns}] \leavevmode
\sphinxstylestrong{None} (\sphinxstyleemphasis{None})

\end{description}\end{quote}

\end{fulllineitems}


\end{fulllineitems}



\chapter{Genarris 2.0 Functions}
\label{\detokenize{index:genarris-2-0-functions}}\index{run\_fhi\_aims\_batch() (in module Genarris.evaluation.run\_fhi\_aims)@\spxentry{run\_fhi\_aims\_batch()}\spxextra{in module Genarris.evaluation.run\_fhi\_aims}}

\begin{fulllineitems}
\phantomsection\label{\detokenize{index:Genarris.evaluation.run_fhi_aims.run_fhi_aims_batch}}\pysiglinewithargsret{\sphinxcode{\sphinxupquote{Genarris.evaluation.run\_fhi\_aims.}}\sphinxbfcode{\sphinxupquote{run\_fhi\_aims\_batch}}}{\emph{comm}, \emph{world\_comm}, \emph{MPI\_ANY\_SOURCE}, \emph{num\_replicas}, \emph{inst=None}, \emph{sname=None}, \emph{structure\_dir=None}, \emph{aims\_output\_dir=None}, \emph{output\_dir=None}, \emph{aims\_lib\_dir=None}, \emph{control\_path=None}, \emph{energy\_name='energy'}, \emph{verbose=False}}{}
Performs multiple FHI calculations
\begin{quote}\begin{description}
\item[{Parameters}] \leavevmode\begin{itemize}
\item {} 
\sphinxstylestrong{comm} (\sphinxstyleemphasis{mpi4py.MPI object}) -- MPI communicator to pass into aims

\item {} 
\sphinxstylestrong{world\_comm} (\sphinxstyleemphasis{mpi4py.MPI object}) -- World MPI communicator

\item {} 
\sphinxstylestrong{MPI\_ANY\_SOURCE} (\sphinxstyleemphasis{mpi4py.MPI.ANY\_SOURCE}) -- Any source object for communication.

\item {} 
\sphinxstylestrong{num\_replicas} (\sphinxstyleemphasis{int}) -- Number of replicas to perform calculation.

\item {} 
\sphinxstylestrong{inst} (\sphinxstyleemphasis{genarris.core.instruct.Instruct}) -- Config Parser object which contains all the configuration file sections
and options for calculation.

\item {} 
\sphinxstylestrong{sname} (\sphinxstyleemphasis{str}) -- Section name which called run\_fhi\_aims\_batch

\item {} 
\sphinxstylestrong{struct\_dir} (\sphinxstyleemphasis{str}) -- Path to directory of structures to perform calculation.

\item {} 
\sphinxstylestrong{aims\_output\_dir} (\sphinxstyleemphasis{str}) -- Path to the directory where FHI-aims calculations should take place.

\item {} 
\sphinxstylestrong{output\_dir} (\sphinxstyleemphasis{str}) -- Path to the directory where the Structure files should be saved.

\item {} 
\sphinxstylestrong{aims\_lib\_dir} (\sphinxstyleemphasis{str}) -- Path to the directory containing the FHI-aims library file.

\item {} 
\sphinxstylestrong{control\_path} (\sphinxstyleemphasis{str}) -- Path to the directory containing the control file to use.

\item {} 
\sphinxstylestrong{energy\_name} (\sphinxstyleemphasis{str}) -- Property name which the calculated energy will be stored with in the
Structure file.

\item {} 
\sphinxstylestrong{verbose} (\sphinxstyleemphasis{bool}) -- Controls verbosity of output.

\end{itemize}

\end{description}\end{quote}
\subsubsection*{Configuration File Options}
\begin{description}
\item[{verbose}] \leavevmode{[}bool{]}
Controls verbosity of output.

\item[{energy\_name}] \leavevmode{[}str{]}
Property name which the calculated energy will be stored with in the
Structure file.

\item[{output\_dir}] \leavevmode{[}str{]}
Path to the directory where the output structure file will be saved.

\item[{aims\_output\_dir}] \leavevmode{[}str{]}
Path where the aims calculation will take place.

\item[{aims\_lib\_dir}] \leavevmode{[}str{]}
Path to the location of the directory containing the FHI-aims library
file.

\item[{molecule\_path}] \leavevmode{[}str{]}
Path to the geometry.in file of the molecule to be calculated if
called using harris\_single\_molecule\_prep or relax\_single\_molecule.

\item[{structure\_dir}] \leavevmode{[}str{]}
Path to the directory of structures to be calculated if calculation
was called not using harris\_single\_molecule\_prep or
relax\_single\_molecule.

\end{description}
\begin{quote}\begin{description}
\item[{Returns}] \leavevmode
\sphinxstylestrong{None} (\sphinxstyleemphasis{None})

\end{description}\end{quote}

\end{fulllineitems}




\renewcommand{\indexname}{Index}
\printindex
\end{document}